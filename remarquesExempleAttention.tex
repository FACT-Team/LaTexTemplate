%	THEOREM STYLES
%----------------------------------------------------------------------------------------

\usepackage{amsmath,amsfonts,amssymb,amsthm} % For including math equations, theorems, symbols, etc

\newcommand{\intoo}[2]{\mathopen{]}#1\,;#2\mathclose{[}}
\newcommand{\ud}{\mathop{\mathrm{{}d}}\mathopen{}}
\newcommand{\intff}[2]{\mathopen{[}#1\,;#2\mathclose{]}}
\newtheorem{notation}{Notation}[chapter]

\newtheoremstyle{ocrenum} % Theorem style name
{7pt} % Space above
{7pt} % Space below
{\normalfont} % Body font
{} % Indent amount
{\small\bf\sffamily\color{ocre}} % Theorem head font
{\;\;} % Punctuation after theorem head
{0.25em} % Space after theorem head
{\small\sffamily\color{ocre}\thmname{#1}\thmnumber{\@ifnotempty{#1}{ }\@upn{#2}} % Theorem text (e.g. Theorem 2.1)
\thmnote{\ {\the\thm@notefont\sffamily\bfseries\color{black}--- #3.}}} % Optional theorem note
\renewcommand{\qedsymbol}{$\blacksquare$} % Optional qed square

\newtheoremstyle{blacknumex} % Theorem style name
{7pt} % Space above
{7pt} % Space below
{\normalfont} % Body font
{} % Indent amount
{\small\bf\sffamily} % Theorem head font
{\;\;} % Punctuation after theorem head
{0.25em} % Space after theorem head
{\small\sffamily{\tiny\ensuremath{\blacksquare}}\ \thmname{#1}\thmnumber{\@ifnotempty{#1}{ }\@upn{#2}} % Theorem text (e.g. Theorem 2.1)
\thmnote{\ {\the\thm@notefont\sffamily\bfseries--- #3.}}} % Optional theorem note

\newtheoremstyle{blacknum} % Theorem style name
{7pt} % Space above
{7pt} % Space below
{\normalfont} % Body font
{} % Indent amount
{\small\bf\sffamily} % Theorem head font
{\;\;} % Punctuation after theorem head
{0.25em} % Space after theorem head
{} % Optional theorem note
\makeatother

% Defines the theorem text style for each type of theorem to one of the three styles above
\theoremstyle{ocrenum}
\newtheorem{theoremeT}{Theorem}[chapter]
\newtheorem{proposition}{Proposition}[chapter]
\newtheorem{problem}{Problem}[chapter]
\newtheorem{attentionT}{}[chapter]
\theoremstyle{blacknum}
\newtheorem{exampleT}{Example}[chapter]
\newtheorem{vocabulary}{Vocabulary}[chapter]
\newtheorem{definitionT}{Definition}[chapter]
\newtheorem{exempleT}{}[chapter]




%----------------------------------------------------------------------------------------
%	DEFINITION OF COLORED BOXES
%----------------------------------------------------------------------------------------

\RequirePackage[framemethod=default]{mdframed} % Required for creating the theorem, definition, exercise and corollary boxes

% Theorem box
\newmdenv[skipabove=7pt,
skipbelow=7pt,
backgroundcolor=black!5,
linecolor=ocre,
innerleftmargin=5pt,
innerrightmargin=5pt,
innertopmargin=5pt,
leftmargin=0cm,
rightmargin=0cm,
innerbottommargin=5pt]{tBox}

% Exercise box	  
\newmdenv[skipabove=7pt,
skipbelow=7pt,
rightline=false,
leftline=true,
topline=false,
bottomline=false,
backgroundcolor=ocre!10,
linecolor=ocre,
innerleftmargin=5pt,
innerrightmargin=5pt,
innertopmargin=5pt,
innerbottommargin=5pt,
leftmargin=0cm,
rightmargin=0cm,
linewidth=4pt]{eBox}	

% Definition box
\newmdenv[skipabove=10pt,
skipbelow=10pt,
rightline=false,
leftline=true,
topline=false,
bottomline=false,
linecolor=ocre,
innerleftmargin=5pt,
innerrightmargin=5pt,
innertopmargin=0pt,
leftmargin=0cm,
rightmargin=0cm,
linewidth=4pt,
innerbottommargin=0pt]{dBox}	

% Corollary box
\newmdenv[skipabove=7pt,
skipbelow=7pt,
rightline=false,
leftline=true,
topline=false,
bottomline=false,
linecolor=gray,
backgroundcolor=black!5,
innerleftmargin=5pt,
innerrightmargin=5pt,
innertopmargin=5pt,
leftmargin=0cm,
rightmargin=0cm,
linewidth=4pt,
innerbottommargin=5pt]{cBox}		

% Corollary box
\newmdenv[skipabove=7pt,
skipbelow=7pt,
rightline=true,
leftline=false,
topline=false,
bottomline=true,
linecolor=gray,
backgroundcolor=black!5,
innerleftmargin=5pt,
innerrightmargin=5pt,
innertopmargin=5pt,
leftmargin=0cm,
rightmargin=0cm,
linewidth=1pt,
innerbottommargin=5pt]{rBox}				  
		  

% Creates an environment for each type of theorem and assigns it a theorem text style from the "Theorem Styles" section above and a colored box from above
\newenvironment{theorem}{\begin{tBox}\begin{theoremeT}}{\end{theoremeT}\end{tBox}}
\newenvironment{example}{\begin{exampleT}}{\hfill{\tiny\ensuremath{\blacksquare}}\end{exampleT}}
\newenvironment{definition}{\begin{dBox}\begin{definitionT}}{\end{definitionT}\end{dBox}}
\newenvironment{attention}{\begin{eBox}\small}{\end{eBox}}				  	
\newenvironment{exemple}{\begin{cBox}\small}{\end{cBox}}	

%----------------------------------------------------------------------------------------
%	REMARK ENVIRONMENT
%----------------------------------------------------------------------------------------

\newenvironment{remarque}{\par\vskip10pt\small
\begin{rBox}
\begin{list}{}{
\leftmargin=35pt % Indentation on the left
\rightmargin=25pt}\item\ignorespaces % Indentation on the right
\makebox[-2.5pt]{\begin{tikzpicture}[overlay]
\node[draw=ocre!60,line width=1pt,circle,fill=ocre!25,font=\sffamily\bfseries,inner sep=2pt,outer sep=0pt] at (-15pt,0pt){\textcolor{ocre}{R}};\end{tikzpicture}} % Orange R in a circle
\advance\baselineskip -1pt}
{\end{list}\vskip1mm\end{rBox}\vskip5pt} % Tighter line spacing and white space after remark


